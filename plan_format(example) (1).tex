\documentclass{jsarticle}
\usepackage[dvipdfmx]{graphicx}


\title{ソフトウェア設計及び実験\\後期・グループ開発企画書 \\10班
	\huge{}
}

\author{6124010964 下浦颯斗}
\date{2025年11月6日}

\begin{document}
\maketitle

\section{ゲーム内容}

\subsection{タイトル}
"Tetra Conflict"

\subsection{ジャンル}
リアルタイム処理系シューティングゲーム(JoyCon(R)使用)​

\subsection{コンセプト・概要}
Joy-Con型個人戦シューティングゲーム

(バトルロワイヤル)


\subsection{世界観}

未定

\subsection{ルール}
Joy-Con(R)を横持ちで操作し自分のプレイヤーを動かして、スティックで狙う向きを定めてXボタンで発砲する。​

制限時間内に他プレイヤーに命中した回数と撃たれた回数で順位を決める。​

武器は4種類の内から好きな武器を選ぶが他人と同じ武器でもよい。​

バトルフィールドには様々なギミックがあるため、敵にばかり気を取られないようにしなくてはならない。

\subsection{UI}
\subsubsection{画面構成}
(記述例)640*480ピクセルのウィンドウは,プレイ部と情報提示部から構成される(図\ref{fig:ui}).プレイ部にステージ,プレイヤキャラクタ,敵キャラクタが表示される.情報提示部には,HP,残り時間とゴールまでの距離が表示される.

\begin{figure}[t]
\begin{center}
\includegraphics[height=10cm]{ui.pdf}
\caption{画面構成}
\label{fig:ui}
\end{center}
\end{figure}

\subsubsection{操作方法}
(記述例)ゲームパッドを用いる.
\begin{itemize}
\item \bf{Aボタン:} ジャンプ
\item \bf{Bボタン:} ポーズ
\item \bf{右ボタン:} 右(+x)方向へ移動
\item \bf{左ボタン:} 左(-x)方向へ移動
\end{itemize}

\section{実装方法}
実装にはSDLを用いる,SDLについては,授業資料\cite{mitsuhara}を参考にする.

\subsection{利用デバイス}
(記述例)ゲームパッドを用いる(図\ref{fig:gp}).SDL\_GameControllerまたはSDL\_Joystickを用いてゲームパッド処理を実装する.

\begin{figure}[t]
\begin{center}
\includegraphics[width=230pt]{gamepad.png}
\caption{ゲームパッド(使用するボタン)}
\label{fig:gp}
\end{center}
\end{figure}

\subsection{データ構造}
(記述例)現時点で必要だと考えられる変数,構造体を示す(表\ref{table1-1})(表\ref{table1-2}).

\begin{table}[tb]
\begin{center}
\caption{データ構造(共通)}
\label{table1-1}
\begin{tabular}{|c||c|l|}\hline
\multicolumn{3}{|l|}{共通の構造}\\ \hline
データ型 & 変数名 &  \\ \hline \hline
int & type & オブジェクトの型(キャラ,アイテム,エフェクト,etc...) \\ \hline
unsigned short & id & オブジェクトの番号 \\ \hline
SDL\_Point & pos & オブジェクトの座標 \\ \hline
int & depth & 深さ(0 or 1) \\ \hline
\end{tabular}
\end{center}
\end{table}

\begin{table}[tb]
\begin{center}
\caption{データ構造(キャラ固有)}
\label{table1-2}
\begin{tabular}{|c||c|c|}\hline
\multicolumn{3}{|l|}{キャラ固有の構造}\\ \hline
データ型 & 変数名 &  \\ \hline \hline
unsigned short & hp & キャラの耐久度 \\ \hline
int & distance & ゴールまでの距離 \\ \hline
char* & name & キャラ名 \\ \hline
\end{tabular}
\end{center}
\end{table}


\subsection{モジュール}
(記述例)現時点で必要だと考えられる関数をモジュールごとに示す(表\ref{table2-1}).

\noindent
(1) ユーザインタフェイス(gui\_process.c)
\par

\begin{table}[tb]
\begin{center}
\caption{ユーザインタフェイス関係関数}
\label{table2-1}
\begin{tabular}{|c||l|}\hline
\multicolumn{2}{|l|}{\bf{void InitWindows(void)}}\\ \hline
関数名 & InitWindows()  \\ \hline \hline
機能 & ウィンドウの初期化処理を行う関数 \\
引数 & なし \\
返り値 & なし \\ \hline
\multicolumn{2}{|l|}{\bf{Bool WindowEvent(void)}}\\ \hline
関数名 & WindowEvent()  \\ \hline \hline
機能 & イベントを取得し,処理する関数 \\
引数 & なし \\
返り値 & 終了フラグ(True:処理継続,False:終了) \\ \hline
\end{tabular}
\end{center}
\end{table}

\section{ガントチャート}
(記述例)1週間ごとのスケジュールを図\ref{fig:gc}に示す.開発に遅れが生じた場合,即座に遅れを取り戻すよう,夜間・休日などは自宅でも開発する.

\begin{figure}[t]
\begin{center}
\includegraphics[scale=0.80]{ganttchart.png}
\caption{開発スケジュールガントチャート}
\label{fig:gc}
\end{center}
\end{figure}

\section{自己評価と理由}
(記述例)表\ref{table3}に示すように,独創性のみ1で,合計値は5となっており,合格基準を満たしていると考える. 

\begin{table}[tb]
\begin{center}
\caption{自己評価と理由}
\label{table3}
\begin{tabular}{|c|c|l|}\hline
項目 & 評価値 & 理由 \\ \hline\hline
ハードウェア構成 & 2 & ゲームパッドを使っているから \\ \hline
技術面 & 2 & 横スクロールアクションゲームで,多数の敵との当たり判定に工夫が必要だから \\ \hline
独創性 & 1 & 見た目がスーパーマリオに似ているから \\ \hline
\end{tabular}
\end{center}
\end{table}

\section{参考にしたゲーム}
(記述例)スーパーマリオブラザーズを参考にした.強制スクロール式にして,とにかく敵が次々出現する.ある種の「避けゲー」になっている点が異なる.

\section{ゲームのアピールポイント}
(記述例)やりこむとほど,うまく敵をよけられるようになりステージクリアとなる中毒性の高いゲーム.5ステージはそれぞれ異なる雰囲気の画像で彩る.

\begin{thebibliography}{9}
\bibitem{mitsuhara} 光原弘幸: ゲーム開発のためのSDL プログラミング-ユーザインタフェイスを中心として,  ソフト実験授業資料 (2019).
\end{thebibliography}

\end{document}






